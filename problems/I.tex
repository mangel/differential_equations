\documentclass{article}

\usepackage{amsmath}
\usepackage{amsfonts}
\usepackage{amsthm}
\usepackage{amssymb}
\usepackage[margin=0.5in]{geometry}
\usepackage[utf8]{inputenc}
\usepackage[spanish, mexico]{babel}

\DeclareMathOperator{\atan}{arctan}

\title{Ejercicios I}
\author{Miguel A. Gomez B.}

\begin{document}
	\maketitle
\paragraph{1.} Verificar que las funciones sean solución de cada una de las ecuaciones diferenciales.

\paragraph{a.} $\frac{dy}{dt} + 20y = 24$, con $y = \frac{6}{5} - \frac{6}{5} e^{-20t}$.

\paragraph{Solución.}

$$y' = 24e^{-20t},$$
reemplazamos en la ecuacion diferencial para verificar:
\begin{align*}
	\frac{dy}{dt} + 20y &= 24e^{-20t} + 20\left(\frac{6}{5} - \frac{6}{5} e^{-20t}\right)\\
	&= 24e^{-20t} + 24 - 24e^{-20t}\\
	&= 24 
\end{align*}
por lo tanto $y$ es solución.
\paragraph{b.} $y'' - 6y' + 13y = 0$, con $y = e^{3x} \cos{(2x)}$.
$$y' = 3e^{2x}\cos{(2x)} - 2e^{2x}\sin{(2x)}$$
y
\begin{align*}
	y'' &= 9e^{3x}\cos{(2x)} - 6e^{3x}\sin{(2x)} - 6e^{3x}\sin{(2x)} - 4e^{3x}\cos{(2x)}\\
	&= 5e^{3x}\cos{(2x)} - 12e^{3x}\sin{(2x)}
\end{align*}
reemplazamos en la ecuación diferencial para verificar en la ecuación original:
$$y'' - 6y' + 13y = 0$$
operando obtenemos los términos que posteriormente sumamos:

\[
	\begin{array}{c c}
		  5e^{3x}\cos{(2x)}   & - 12e^{3x}\sin{(2x)} \\
		- 18e^{3x}\cos{(2x)} & + 12e^{3x}\sin{(2x)} \\
		+ 38e^{3x}\cos{(2x)} &\\
		\hline
		\multicolumn{2}{c}{0}
	\end{array}
\]
Vemos por lo tanto que $y$ es solución de la ecuación diferencial.
\paragraph{c.} $y'' + y = \tan{(x}$, con $y = -(\cos{(x)}) \ln{(\sec{(x)} + \tan{(x)})}$.
\begin{align*}
	y' &= \sin{(x)} \ln{(\sec{(x)} + \tan{(x)})} - \cos{(x)} \sec{(x)}\\
	&= \sin{(x)} \ln{(\sec{(x)} + \tan{(x)})} - 1
\end{align*}
y
\begin{align*}
	y'' &= \cos{(x)} \ln{(\sec{(x)} + \tan{(x)})} + \sin{(x)}\sec{(x)}\\
	&= \cos{(x)} \ln{(\sec{(x)} + \tan{(x)})} + \tan{(x)}
\end{align*}
Verificamos:
\[
	\begin{array}{l l}
		  \cos{(x)}\ln{(\sec{(x)} + \tan{(x)})} &  + \tan{(x)}\\
		- \cos{(x)}\ln{(\sec{(x)} + \tan{(x)})} & \space\\
		\hline
		\multicolumn{2}{r}{\tan{(x)}}
	\end{array}
\]
Vemos por tanto que $y$ es una solución.
\paragraph{2.} Solucione las ecuaciones diferenciales de manera analítica.
\paragraph{a.} $\frac{dQ}{dt} = k(Q - 70)$.
\paragraph{Solución.} Utilizamos el método de variable separable.
$$\frac{dQ}{dt} = k(Q - 70) \Rightarrow \frac{1}{k(Q-70)} dQ = dt$$
Integrando, tenemos por separado que:
\begin{align*}
	\int \frac{1}{k(Q-70)} dQ &= \ln{(k(Q - 70))} + C_1
\end{align*}
\begin{align*}
	\int dt = t + C_2
\end{align*}
de modo que ahora tenemos
$$\ln{(k(Q - 70))} + C_1 = t + C_2 \Rightarrow \ln{(k(Q - 70))} = t + C_3$$
con $C_3 = C_2 - C_1 $, utilizamos la función exponencial para hallar la inversa del logaritmo y despejar $Q$:
\begin{align*}
	k(Q - 70) &= e^{t + C_3} = e^t e^{C_3} = e^t C_4\\
	Q &= \frac{e^t C_4 + 70}{k}
\end{align*}
con $C_4 = e^{C_3}$.
\paragraph{} \textit{A partir de éste punto se asumirá C como la suma de las constantes de integración.}
\paragraph{b.} $\frac{dy}{dx} = e^{3x + 2y}$.
\paragraph{Solución.} Utilizamos el método de variable separable. Nótese que
$e^{3x +2y} = e^{3x}e^{2y}$. Luego tenemos:
$$\frac{dy}{dx} = e^{3x + 2y} = e^{3x} e^{2y},$$
Realizando la separación de variables obtenemos:
$$\frac{1}{e^{2y}} dy = e^{3x} dx,$$
de modo que ahora integramos por separado,
$$\int \frac{1}{e^{2y}} dy = -\frac{1}{2e^{2y}} + Q$$
y
$$\int e^{3x} dx = \frac{1}{3} e^{3x} + P,$$
agrupando los dos resultados tenemos la igualdad
$$-\frac{1}{2e^{2y}} = \frac{1}{3} e^{3x} + C$$
Reorganizamos para despejar $y$,
\begin{align*}
	-\frac{1}{2e^{2y}} &= \frac{1}{3} e^{3x} + C\\
	e^{2y} &= -\frac{3}{2e^{3x}} + C_1\\
	2y &= e^{-\frac{3}{2e^{3x}}} C_2\\
	y &= \frac{1}{2} e^{-\frac{3}{2e^{3x}}} C_2
\end{align*}
con $C_1 = \frac{1}{2C}$ y $C_2 = e^{C_1}$.
\paragraph{c.} $\sin{(x)} dx + 2y \cos^{3}{(3x)} dy = 0$.
\paragraph{Solución.} Reorganizamos la expresión para poder aplicar el método de variable separable,
\begin{align*}
	\sin{(x)} dx + 2y \cos^{3}{(3x)} dy &= 0\\
	\frac{\sin{(3x)}}{\cos^{3}{3x}} dx &= -2y dy\\
	\frac{\tan{(3x)}}{\cos^2{3x}} dx &= \\
	\tan{(3x)} \sec^2{(3x)} &= -2ydy,
\end{align*}
Ahora integramos y reorganizamos la expresión para despejar en terminos de x,
\begin{align*}
	\int \tan{(3x)} \sec^2{(3x)} &= \int -2ydy\\
	\frac{1}{6} \tan^2{3x} &= -y^2 + C\\
	\tan^2{(3x)} &= 6(-y^2 + C)\\
	\tan{(3x)} &= \sqrt{6(-y^2 + C)}\\
	3x &= \atan{(\sqrt{6(-y^2 + C)})}\\
	x &= \frac{1}{3} \atan{(\sqrt{6(-y^2 + C)})}.
\end{align*}
\paragraph{d.} $\frac{dy}{dx} = \left(\frac{2y + 3}{4x + 5}\right)^2$.
\paragraph{Solución.} Reorganizamos la expresión,
\begin{align*}
	\frac{dy}{dx} &= \left(\frac{2y + 3}{4x + 5}\right)^2\\
	\frac{1}{(2y + 3)^2}dy &= \frac{1}{(4x + 5)^2}dx,
\end{align*}
Integramos. Para ello utilizamos el método de sustitución, para el extremo izquierdo sustituimos $u = 2y + 3$ y en el derecho $v = 4x + 5$ y tenemos
\begin{align*}
	\frac{1}{2}\int \frac{1}{u^2}dy &= \frac{1}{4} \int \frac{1}{v^2}dv\\
	\frac{1}{2}\int u^{-2}du &= \frac{1}{4} \int v^{-2} dv\\
	- u^{-1} &= -\frac{1}{2} v^{-1} + C \\
	u^{-1} &= \frac{1}{2} v^{-1} + C \\
	u &= 2v + C^{-1} \\
	2y + 3 &= 2(4x + 5) + C^{-1}\\
	y &= \frac{1}{2}(8x + 10 - 3 + C^{-1})\\
	y  &= \frac{1}{2}(8x + 7 + C^{-1}).\\
\end{align*}

\paragraph{e.} $\frac{dP}{dt} = P - P^{2}$.
\paragraph{Solución.} Reorganizamos la expresión,
\begin{align*}
	\frac{dP}{dt} &= P - P^{2}\\
	\frac{1}{P - P^{2}} dP &= dt\\
	\frac{1}{P(1 - P)} dP &= dt
\end{align*}
ahora integramos la expresión,
$$\int \frac{1}{P(1 - P)} dP = \int dt$$
para ello utilizaremos el método de fracciones parciales, en el extremo izquierdo.
\begin{align*}
-\frac{1}{P(P-1)} &= -\frac{A}{P} - \frac{B}{P-1}\\
1 &= A(P-1) + BP,
\end{align*}
si $P=0$, entonces $A=-1$. Si $P=1$, entonces $B = 1$ luego nuestra integral final toma la forma
\begin{align*}
\int \left( \frac{1}{P} - \frac{1}{P-1}\right) dP &= \int dt\\
	\ln{(P)} - \ln{(P-1)} &= t + C\\
	\ln{\left(\frac{P}{P-1}\right)} &= t + C\\
	\frac{P}{P-1} &= e^{t+C}\\
	P &= e^{t+C}(P-1)\\
	P &= e^{t+C}P - e^{t+C}\\
	P - e^{t+C}P &= - e^{t+C}\\
	P(1 - e^{t+C}) &= - e^{t+C}\\
	P &= - \frac{e^{t+C}}{1 - e^{t+C}}
\end{align*}
\end{document}